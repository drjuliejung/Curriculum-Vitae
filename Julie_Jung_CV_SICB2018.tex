\documentclass[margin,line]{res}

\usepackage{hyperref}

\oddsidemargin -.5in
\evensidemargin -.5in
\textwidth=6.0in
\itemsep=0in
\parsep=0in
% if using pdflatex:
%\setlength{\pdfpagewidth}{\paperwidth}
%\setlength{\pdfpageheight}{\paperheight} 

\newenvironment{list1}{
  \begin{list}{\ding{113}}{%
      \setlength{\itemsep}{0in}
      \setlength{\parsep}{0in} \setlength{\parskip}{0in}
      \setlength{\topsep}{0in} \setlength{\partopsep}{0in} 
      \setlength{\leftmargin}{0.17in}}}{\end{list}}
\newenvironment{list2}{
  \begin{list}{$\bullet$}{%
      \setlength{\itemsep}{0in}
      \setlength{\parsep}{0in} \setlength{\parskip}{0in}
      \setlength{\topsep}{0in} \setlength{\partopsep}{0in} 
      \setlength{\leftmargin}{0.2in}}}{\end{list}}

\usepackage{setspace}
\renewcommand{\baselinestretch}{1.0} 


\begin{document}

\name{Julie Jung \vspace*{.1in}}

\begin{resume}
\section{\sc Contact Information}
\vspace{.05in}
\begin{tabular}{@{}p{2in}p{4in}}
36 Long Avenue & {\it Cell:}  (510) 439-7614 \\  
Apartment \#1 & {\it E-mail:}  {\ttfamily jungjulie2@gmail.com} \\

Allston, MA 02134 & {\it Website:} \href{http://jungjulie.wordpress.com}{\ttfamily http://jungjulie.wordpress.com}\\       
& {\it Github:} \href{https://github.com/jamjulie}{\ttfamily https://github.com/jamjulie}\\     

\end{tabular}

\section{\sc Education}
{\bf Boston University}, Boston, MA (Cumulative GPA: 3.91)

\vspace*{-4mm}
Dual M.A./Ph.D. in Biology, June 2015 - present\\
Dissertation: {\it The Ontogeny of Vibration-cued Early Hatching in Red-Eyed Treefrogs}\\ 
Advisors: Drs. Karen M. Warkentin (Biology) \& James G. McDaniel (Mechanical Engineering)

{\bf Williams College}, Williamstown, MA (Cumulative GPA: 3.42, Jr/Sr GPA: 3.52)

\vspace*{-4mm}
B.A. {\it with Honors} in Biology \& Environmental Science, Minor in Maritime Studies, June 2015 \\
Senior Biology Honors Thesis: {\it The Influence of Land Management Practices on the Abundance and Diversity of Fall-Blooming Asteraceae and Their Pollinators}\\
Advisor: Dr. Joan Edwards (Biology)


{\bf The College Preparatory School}, Oakland, CA (Cumulative GPA: 3.90)

\vspace*{-4mm}
Student Body Treasurer, June 2011 

\section{\sc Testing} 
{\bf GRE:} 324/340, 
{\bf SAT:} 2330/2400

\section{\sc Honors, Awards, and Grants} 

{\bf Biology Department Travel Award}, Boston University, 2017 (\$400)
\vspace*{-3.5mm}

{\bf Charlotte  Magnum	Student	Support	Scholarship}, SICB, 2016 \& 2107
\vspace*{-3.5mm}

{\bf NSF GRFP Honorable Mention}, 2016 
\vspace*{-3.5mm}

{\bf Williams College Biology Conference Travel Award}, 2015 (\$500)
\vspace*{-3.5mm}

{\bf Tom Hardie Prize in Environmental Studies}, Williams College, 2015 (\$500)
\vspace*{-3.5mm}

{\bf NSF REU, Cary Institute of Ecosystem Studies}, 2014 (\$7600)
\vspace*{-3.5mm}

{\bf Environmental Studies Department Class of 1960 Scholar}, Williams College, 2014-15
\vspace*{-3.5mm}

{\bf Dean's List}, Williams College, Fall 2013 - Spring 2015 (all semesters)
\vspace*{-3.5mm}

{\bf Steel Family Scholarship for Teaching}, Williams College, 2011-15 (\$212,000)
\vspace*{-3.5mm}

{\bf Seoul National University Scholarship}, Williams College, 2011 (\$2,000)
\vspace*{-3.5mm}

{\bf National Merit Commendation}, The College Preparatory School, 2011


\section{\sc Skills} 
{\bf Computing Languages}:  R (fluent), Matlab (highly proficient), Python (proficient).  
\vspace*{-3mm}

{\bf Other Computing Skills}: \LaTeX, Git \& Github, MS Excel \& Powerpoint, Adobe Photoshop \& Illustrator CS6, SYSTAT, Prism, Raven, Audacity. 
\vspace*{-3mm}

{\bf Certifications}: SCUBA, Red Cross CPR, First Aid, Lifeguard. 
\vspace*{-3mm}

{\bf Languages}: Korean (fluent), English (fluent), Spanish (highly proficient)

% \section{\sc Research Interests}
% {\bf Behavioral Ecology} Optimal decision theory, optimal foraging theory, bioacoustics/vibrations, vestibular systems, phenotypic plasticity, automated behavioral assays. 
% 
% \vspace{-.3cm}
% {\bf Developmental Biology} Neuroimaging, confocal, micro-CT, embryo morphology and histology, in-situ hybridization, transgenic techniques, cloning and cell culture. 

\section{\sc Publications}

Warkentin K.M., {\bf J. Jung}, L.A.R. Solano, J.G. McDaniel. 2017. {Ontogeny of escape-hatching decisions: discrimination among vibrational cues changes developmentally as predicted from costs of sampling and false alarms.} (Under review in {\it Behavioral Ecology}.) 

\vspace{-.3cm}
Warkentin K.M., J.C. Diaz, B.A. Guell, {\bf J. Jung}, S.J. Kim, K.L. Cohen. 2017. {Developmental onset of the escape-hatching response in red-eyed treefrogs depends on cue type.} {\it Animal Behaviour.} {129:103-112.} \href{http://www.sciencedirect.com/science/article/pii/S0003347217301458}{\ttfamily https://doi.org/10.1016/j.anbehav.2017.05.008}

\vspace{-.3cm}
{\bf Jung J.} and K.A. Schmidt. 2014. {Anthropogenic noise: The effects of road noise on eavesdropping systems of the eastern chipmunk.} \href{http://www.caryinstitute.org/sites/default/files/public/reprints/jung_2014_REU.pdf}{\ttfamily {\it Undergraduate Ecology Research Reports}} (and {\it in preparation} for submission to {\it Israel Journal of Ecology and Evolution}).
  


\section{\sc Manuscripts in Preparation}
{\bf Jung J.}, S.J. Kim, S.P. Arias, J.G. McDaniel, K.M. Warkentin. {How do red-eyed treefrog embryos detect snake attacks? Assessing the role of vestibular mechanoreception.} ({\it in preparation} for submission to {\it Journal of Experimental Biology}.)

\vspace{-.3cm}
{\bf Jung J.}, J.G. McDaniel, K.M. Warkentin. {Ontogenetic adaptation in information use for escape-hatching decisions: older embryos selectively accept more false alarms.} ({\it in preparation} for submission to {\it Behavioral Ecology}.)

\vspace{-.3cm}
{\bf Jung J.} and K.M. Warkentin. {Inner ear development across onset and improvement of escape-hatching ability in red-eyed treefrogs: a confocal and µCT analysis.} ({\it in preparation} for submission to {\it Journal of Experimental Biology}.)

\section{\sc Conference Presentations}

Edwards, J., {\bf J. Jung}, L. Davis, and D. Smith. {2017.} {The Influence of Land Management Practices on the Abundance and Diversity of Fall-Blooming Asteraceae and Their Pollinators.} {\it Entomological Society of America Meeting,} {Denver, CO.}

\vspace{-.25cm}
{\bf Jung J.}, J.G. McDaniel, K.M. Warkentin. {2017.} {Ontogeny of vibration-cued escape-hatching in red-eyed treefrogs: two reasons older embryos hatch more.} {\it Society for Integrative and Comparative Biology Meeting,} {New Orleans, LA.}

\vspace{-.25cm}
Kim, S.J., {\bf J. Jung}, S.M. Pérez Arias, J.G. McDaniel, K.M. Warkentin. {2016.} {Is ear function necessary for vibration-cued hatching in red-eyed treefrogs?} {\it Animal Behavior Society Meeting} {, Colombia, MO.}

\vspace{-.25cm}

{\bf Jung J.}, S.J. Kim, B.A. Guell, K.L. Cohen, K.M. Warkentin. {2016.} {Ontogeny of escape hatching in red-eyed treefrogs: onset of response to flooding and attack cues.} {\it Society for Integrative and Comparative Biology Meeting,} {Portland, OR.}

\vspace{-.25cm}
Kim, S.J., {\bf J. Jung}, S.M. Pérez Arias, J.G. McDaniel, K.M. Warkentin. {2016.} {Shake and roll: testing the ontogenetic correlation of vibration-cued hatching and otic mechanoreception in red-eyed treefrogs.} {\it Society for Integrative and Comparative Biology Meeting,} {Portland, OR.}

\vspace{-.25cm}
Warkentin, K.M., Cohen, K.L., Diaz, J.C., Guell, B.A., and {\bf J. Jung}. {2016.} {Development of embryo behavior: Hatching mechanisms, performance, and decisions in red-eyed treefrogs.} {\it Society for Integrative and Comparative Biology Meeting,} {Portland, OR.}

\vspace{-.25cm}
Perez, D.J., {\bf J. Jung}, K.A. Schmidt. {2015.} {Anthropogenic noise: The effects of road noise on eavesdropping systems of the eastern chipmunk} {\it Ecological Society of America,} {Baltimore, MD.}

\vspace{-.25cm}
{\bf Jung J.} and K.A. Schmidt. {2015.} {Consider the chipmunk: road noise effects on eavesdropping systems in eastern chipmunks.} {\it Emory University Laney Graduate School STEM Symposium,} {Atlanta, GA. }

\section{\sc Teaching / Curriculum Development Experience}
{\bf Teaching Fellow},  Boston University, 8/15 - 12/16 

\vspace{-.43cm}
{1 semester for BI302 Vertebrate Zoology; 1 semester for BI107 Ecology and Evolution. Positions involved lecturing to 2 weekly lab sections of 25-30 students from 1-3 hours, conducting long-term field studies, grading weekly homework assignments, lab reports, and exams. }

\vspace{-.1cm}
{\bf Teaching Assistant},  Williams College, 2/14-6/15 

\vspace{-.43cm}
{2 semesters for ENVI102 Introduction to Environmental Science; 1 semester for BIOL203 Ecology; 1 semester for MATH103 Calculus. Positions had varying levels of involvement - from making lesson plans, preparing quizzes, and grading homework assignments to driving students to field sites and inputting lab data.}

\vspace{-.1cm}
{\bf Intern},  TERC in Cambridge, MA, 12/13-2/14

\vspace{-.43cm}
{Helped to develop a high school capstone course in Ecological Environmental Science, focusing on curricula materials involving biology and climate-science, as part of the Life Sciences Initiative at TERC, a non-profit organization dedicated to education research and evaluation. }

\vspace{-.1cm}
{\bf Science Teacher},  Greylock Elementary School in North Adams, MA, 9/11-12/13

\vspace{-.43cm}
{Encouraged hands-on science learning in a classroom of 17 fifth graders through the Williams Elementary Outreach Program. Worked closely with classroom teachers to run weekly hour-long lessons on adaptation. }

\section{\sc Research Experience}
{\bf Dissertation Research},  Boston University, 6/15 - Expected 2020 

\vspace{-.43cm}
{Advisors: Professors Karen Warkentin (Biology) and J. Greg McDaniel (Mechanical Engineering). Project: vibration-cued early-hatching behaviors in red-eyed treefrog embryos. }

\vspace{-.1cm}
{\bf Honors Thesis Research},  Williams College Department of Biology, 8/14-6/15 

\vspace{-.43cm}
{Advisor: Professor Joan Edwards. Project: (i) Constructed spatial distribution maps showing the density and diversity of ever species within the study area. (ii) Set out timelapse videos to capture and analyze pollination events on select stems.} 

\vspace{-.1cm}
{\bf REU},  Cary Institute of Ecosystem Studies in Millbrook, NY, 5/14-8/14

\vspace{-.43cm}
{Advisor: Professor Kenneth Schmidt. Project: (i) Recorded and edited chipmunk, titmouse, and veery vocalizations. (ii) Designed, set up, and conducted giving-up density and playback experiments examining road noise effects on eavesdropping systems in the {\it Tamias striatus-Baeolophus bicolor} dyad.}

\vspace{-.1cm}
{\bf Research Assistant},  Williams College Center for Environmental Studies, 5/12-9/12

\vspace{-.43cm}
{Advisor: Jason Racela. Project: (i) Analyzed samples of local water to test for quality and ion balance. Maintained instruments and databases. (ii) Gained experience with atomic absorption spectroscopy, scanning electron microscopy, ion chromatography.}

\vspace{-.1cm}
{\bf Research Assistant},  US Department of Agriculture in Albany, CA, 9/08-6/11

\vspace{-.43cm}
{Advisor: Dr. Susan B. Altenbach. Project: (i) Helped design an RNAi construct to silence the expression of genes that trigger allergies to US bread wheat Butte 86. (ii) Dissected wheat embryos. (iii) Used PCR to confirm stable transformation and inheritance of transgenes in embryo samples. (iv) Maintained greenhouses.}

\section{\sc Society Memberships; Service and Outreach}

{\bf American Association for the Advancement of Science}, Member, 2016 - Present 

{\bf Society for Integrative and Comparative Biology}, Graduate Student Member, 2015 - Present 

{\bf Sigma Xi},  Associate Member, 2015 - 2016 

{\bf Volunteer teacher},  "BIOBUGS" program for high-school students, 9/15 - Present 

{\bf Laboratory Safety Coordinator},  Warkentin Lab, Boston University, 1/15 - Present 

{\bf Research Experiences for Teachers},  Collaboration with high school teachers to develop lessons for secondary students based on red-eyed treefrog research, 6/15 - Present

\section{\sc Undergraduate Students Mentored}

{\bf Su Jin Kim}, UROP student

{\bf Adeline Almanzar}, volunteer, UROP student

{\bf Alina Chaiyasarikul}, volunteer, UROP student

\end{resume}

\end{document}
$\]
\end{document}