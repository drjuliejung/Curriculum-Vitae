\documentclass[margin,line]{res}

\usepackage{hyperref}

\oddsidemargin -.5in
\evensidemargin -.5in
\textwidth=6.0in
\itemsep=0in
\parsep=0in
% if using pdflatex:
%\setlength{\pdfpagewidth}{\paperwidth}
%\setlength{\pdfpageheight}{\paperheight} 

\newenvironment{list1}{
  \begin{list}{\ding{113}}{%
      \setlength{\itemsep}{0in}
      \setlength{\parsep}{0in} \setlength{\parskip}{0in}
      \setlength{\topsep}{0in} \setlength{\partopsep}{0in} 
      \setlength{\leftmargin}{0.17in}}}{\end{list}}
\newenvironment{list2}{
  \begin{list}{$\bullet$}{%
      \setlength{\itemsep}{0in}
      \setlength{\parsep}{0in} \setlength{\parskip}{0in}
      \setlength{\topsep}{0in} \setlength{\partopsep}{0in} 
      \setlength{\leftmargin}{0.2in}}}{\end{list}}

\usepackage{setspace}
\renewcommand{\baselinestretch}{1.0} 


\begin{document}

\name{Julie Jung \vspace*{.1in}}

\begin{resume}
\section{\sc Contact Information}
\vspace{.05in}
\begin{tabular}{@{}p{2in}p{4in}}
36 Long Avenue & {\it Cell:}  (510) 439-7614 \\  
Apartment \#1 & {\it E-mail:}  {\ttfamily jungj@bu.edu} \\

Allston, MA 02134 & {\it Website:} \href{http://jungj.blogspot.com}{\ttfamily http://jungj.blogspot.com}\\       
& {\it Github:} \href{https://github.com/jamjulie}{\ttfamily https://github.com/jamjulie}\\     

\end{tabular}

\section{\sc Education}
{\bf Boston University}, Boston, MA

\vspace*{-4mm}
Ph.D. in Biology, Fall 2015-present

{\bf Williams College}, Williamstown, MA

\vspace*{-4mm}
B.A.  {\it honors} Biology \& Environmental Science, Minor in Maritime Studies, June 2015 \\
Senior Biology Honors Thesis: {\it The Influence of Land Management Practices on the Abundance and Diversity of Fall-Blooming Asteraceae and Their Pollinators} 


{\bf The College Preparatory School}, Oakland, CA

\vspace*{-4mm}
Student Body Treasurer, June 2011

\section{\sc Honors and Awards} 

{\bf NSF GRFP Honorable Mention}, 2016

\vspace*{-3.5mm}
{\bf Tom Hardie Prize in Environmental Studies}, Williams College, 2015

\vspace*{-3.5mm}
{\bf Environmental Studies Department Class of 1960 Scholar}, Williams College, 2014-15

\vspace*{-3.5mm}
{\bf Dean's List}, Williams College, Fall 2013 - Spring 2015 (all semesters)

\vspace*{-3.5mm}
{\bf Steel Family Scholarship for Teaching}, Williams College, 2013-14

\vspace*{-3.5mm}
{\bf Seoul National University Scholarship}, Williams College, 2011

\vspace*{-3.5mm}
{\bf National Merit Commendation}, The College Preparatory School, 2011

\section{\sc Skills} 
{\bf Computing Languages and Packages}:  Highly proficient in R \& Matlab, limited exposure to Python. 
\vspace*{-3mm}

{\bf Other Computing Skills}: \LaTeX, Github, MS Excel \& Powerpoint, Adobe Photoshop \& Illustrator CS6, ArcGIS, SYSTAT
\vspace*{-3mm}

{\bf Certifications}: SCUBA, Red Cross CPR, First Aid, Lifeguard. 
\vspace*{-3mm}

{\bf Languages}: Korean (fluent), English (fluent), Spanish (proficient)

\section{\sc Research Interests}
{\bf Behavioral Ecology} Optimal decision theory, optimal foraging theory, vibrational cues, predator-prey interactions, phenotypic plasticity, tropical field biology. 

\vspace{-.3cm}
{\bf Developmental Biology} Confocal imaging, Micro-CT imaging, morphology, histology. 

\section{\sc Manuscripts in Preparation}

Warkentin K.M., J.C. Diaz, B.A. Guell, {\bf J. Jung}, S.J. Kim, K.L. Cohen. {Developmental onset of the escape-hatching response in red-eyed treefrogs depends on cue type.} ({\it in preparation}  for submission to {\it Animal Behaviour}.) 

\vspace{-.3cm}
Warkentin K.M., {\bf J. Jung}, L.A.R. Solano, J.G. McDaniel. {Ontogeny of escape-hatching decisions: discrimination among vibrational cues changes developmentally as predicted from costs of sampling and false alarms.} ({\it in preparation}  for submission to {\it Jounal of Experimental Biology}.) 

\vspace{-.3cm}
{\bf Jung J.}, S.J. Kim, S.P. Arias, J.G. McDaniel, K.M. Warkentin. {Shake and roll: testing the ontogenetic correlation of vibration-cued hatching and otic mechanoreception in red- eyed treefrogs.} ({\it in preparation} for submission to {\it Jounal of Experimental Biology}.)  

\vspace{-.3cm}
D.J. Perez, {\bf Jung J.}, K.A. Schmidt. {Anthropogenic noise: The effects of road noise on eavesdropping systems of the eastern chipmunk.} ({\it in preparation}  for submission to {\it Animal Behaviour}.)  

\section{\sc Presentations at Scientific Meetings}

{\bf Jung J.}, J.G. McDaniel, K.M. Warkentin. {\it Pending - Jan. 2017.}. {Ontogeny of vibration-cued escape-hatching in red-eyed treefrogs: two reasons older embryos hatch more.} {\it Society for Integrative and Comparative Biology Meeting,} {New Orleans, LA.}

\vspace{-.25cm}
Kim, S.J., {\bf J. Jung}, S.M. Pérez Arias, J.G. McDaniel, K.M. Warkentin. {2016.} {Is ear function necessary for vibration-cued hatching in red-eyed treefrogs?} {\it Animal Behavior Society Meeting} {, Colombia, MO.}

\vspace{-.25cm}

{\bf Jung J.}, S.J. Kim, B.A. Güell, K.L. Cohen, K.M. Warkentin. {2016.} {Ontogeny of escape hatching in red-eyed treefrogs: onset of response to flooding and attack cues.} {\it Society for Integrative and Comparative Biology Meeting,} {Portland, OR.}

\vspace{-.25cm}
Kim, S.J., {\bf J. Jung}, S.M. Pérez Arias, J.G. McDaniel, K.M. Warkentin. {2016.} {Shake and roll: testing the ontogenetic correlation of vibration-cued hatching and otic mechanoreception in red-eyed treefrogs.} {\it Society for Integrative and Comparative Biology Meeting,} {Portland, OR.}

\vspace{-.25cm}
Perez, D.J., {\bf J. Jung}, K.A. Schmidt. {2015.} {Anthropogenic noise: The effects of road noise on eavesdropping systems of the eastern chipmunk} {\it Ecological Society of America,} {Baltimore, MD.}

\vspace{-.25cm}
{\bf Jung J.} and K.A. Schmidt. {2015.} {Consider the chipmunk: road noise effects on eavesdropping systems in eastern chipmunks.} {\it Emory University Laney Graduate School STEM Symposium,} {Atlanta, GA. }

\section{\sc Research Experience}
{\bf PhD Research},  Boston University, 6/15 - Present 

\vspace{-.43cm}
{Advisors: Professor Karen Warkentin (Biology) and Professor J. Greg McDaniel (Mechanical Engineering). Project: vibration-cued early-hatching behaviors in red-eyed treefrog embryos. }

\vspace{-.1cm}
{\bf Honors Thesis Research},  Williams College Department of Biology, 8/14-6/15 

\vspace{-.43cm}
{Advisor: Professor Joan Edwards. Project: (i) Constructed spatial distribution maps showing the density and diversity of ever species within the study area. (ii) Set out timelapse videos to capture and analyze pollination events on select stems.} 

\vspace{-.1cm}
{\bf REU},  Cary Institute of Ecosystem Studies in Millbrook, NY, 5/14-8/14

\vspace{-.43cm}
{Advisor: Professor Kenneth Schmidt. Project: (i) Recorded and edited chipmunk, titmouse, and veery vocalizations. (ii) Designed, set up, and conducted giving-up density and playback experiments examining road noise effects on eavesdropping systems in the {\it Tamias striatus-Baeolophus bicolor} dyad.}

\vspace{-.1cm}
{\bf Research Assistant},  Williams College Center for Environmental Studies, 5/12-9/12

\vspace{-.43cm}
{Advisor: Jason Racela. Project: (i) Analyzed samples of local water to test for quality and ion balance. Maintained instruments and databases. (ii) Gained experience with atomic absorption spectroscopy, scanning electron microscopy, ion chromatography.}

\vspace{-.1cm}
{\bf Research Assistant},  US Department of Agriculture in Albany, CA, 9/08-6/11

\vspace{-.43cm}
{Advisor: Dr. Susan B. Altenbach. Project: (i) Helped design an RNAi construct to silence the expression of genes that trigger allergies to US bread wheat Butte 86. (ii) Dissected wheat embryos. (iii) Used PCR to confirm stable transformation and inheritance of transgenes in embryo samples. (iv) Maintained greenhouses.}

\section{\sc Teaching Experience / Curriculum Development}
{\bf Teaching Fellow},  Boston University, 8/15 - Present 

\vspace{-.43cm}
{1 semester for BI302 Vertebrate Zoology; 1 semseter for BI107 Ecology and Evolution. Positions involved lecturing to 2 weekly lab sections of 25-30 students from 1-3 hours, conducting long-term field studies, grading weekly homework assignments, lab reports, and exams. }

\vspace{-.1cm}
{\bf Teaching Assistant},  Williams College, 2/14-6/15 

\vspace{-.43cm}
{2 semesters for ENVI102 Introduction to Environmental Science; 1 semester for BIOL203 Ecology; 1 semester for MATH103 Calculus. Positions had varying levels of involvement - from making lesson plans, preparing quizzes, and grading homework assignments to driving students to field sites and inputting lab data.}

\vspace{-.1cm}
{\bf Intern},  TERC in Cambridge, MA, 12/13-2/14

\vspace{-.43cm}
{Helped to develop a high school capstone course in Ecological Environmental Science, focusing on curricula materials involving biology and climate-science, as part of the Life Sciences Initiative at TERC, a non-profit organization dedicated to education research and evaluation. }

\vspace{-.1cm}
{\bf Science Teacher},  Greylock Elementary School in North Adams, MA, 9/11-12/13

\vspace{-.43cm}
{Encouraged hands-on science learning in a classroom of 17 fifth graders through the Williams Elementary Outreach Program. Worked closely with classroom teachers to run weekly hour-long lessons on adaptation. }

\section{\sc Service and Outreach}

{\bf Volunteer teacher},  "BIOBUGS" program for high-school students, 9/15 - Present 

{\bf Laboratory Safety Coorinator},  Warkentin Lab, Boston University, 1/15 - Present 

{\bf Research Experiences for Teachers},  Collaboration with high school teachers to develop lessons for secondary students based on red-eyed treefrog research, 6/15 - Present

\section{\sc Undergraduate Students Mentored}
{\bf Su Jin Kim},  UROP student, Boston University, Summer 2015 - Present

{\bf Alina Chaiyasarikul}, Volunteer and UROP student, Boston University, Summer 2016 - Present

{\bf Adeline Paula Almanzar},  Volunteer, Boston University, Fall 2016 - Present


\end{resume}
\end{document}
$\]
\end{document}